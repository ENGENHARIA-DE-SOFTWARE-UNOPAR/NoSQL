
\section{Introdução}


\noindent \begin{minipage}[c]{0.6\textwidth}
\par O uso de banco e dados teve seu início com o advento da computação, para o armazenar e gerenciar dados, com as necessidades aumentando diversos pesquisadores foram desenvolvendo e aprimorando uma marco importante segundo \cite{histdb:2024}, em meados de 1980 a IBM desenvolve a \textit{Structured Query Language} (SQL), com o passar dos tempo o ícone que represenda um banco de dados é o da figura \ref{fig:icodb}. Seguindo os avanços e nesseciades tecnilogicas temos dois grandes tipos de banco de dados: os \textbf{Relacionais} e os \textbf{Não relacionais}, dos quais o objeto tema desta aula prática seráo \textbf{Não Relacionais}.
\par Esta aula pratica tem por objetivo implementação de um banco de dados \textbf{Não relacional} em um contexto de um loja, a referida atividade será desenvolvida em duas etapas: Primeira atividade \ref{1atividade} e Segunda atividade \ref{2atividae}


\end{minipage}
\begin{minipage}[c]{0.4\textwidth}
  \captionof{figure}{Ícone de banco de dados}
  \includegraphics[width=\textwidth]{figure/pngegg.png}
  \label{fig:icodb}

  {\fontsize{10pt}{\baselineskip}\selectfont
    Fonte: \citeonline{DB:2024}}
\end{minipage}


\section{Métodos}

\par É porposto a elaboração de uma aula prática, tendo como fim a aplicação dos conhecimentos adqueridos em sala de aula virtual, e para tal é disponibilizado um \href{https://github.com/ENGENHARIA-DE-SOFTWARE-UNOPAR/NoSQL/blob/main/Roteiro-aula%20pratica.pdf}{roteiro aula prática}. Utiliza-se o \textit{software} \textbf{MongoDB} \ref{fig:logodb}, o referido \textit{software} é de código aberto e seu código pode ser consultado no \href{https://github.com/mongodb}{repositório do GitHub}.

\begin{figure}[H]
  \caption{Logo MongoDB}
  \includegraphics[scale=0.8]{figure/mongologo.png}
  \label{fig:logodb}

  {\fontsize{10pt}{\baselineskip}\selectfont
    Fonte: \citeonline{mongo:2024}}
\end{figure}




\section{Resultados}
\subsection{Primeira atividade}\label{1atividade}

\subsection{Segunda Atividade}\label{2atividae}

%\lstinputlisting[language=c, caption={código externo}, label={cod:externo}, captionpos=t]{main.c} %Busca os codigos na pasta /cod


\begin{figure}[H] %Figuras da aula pratica 1.1
  \center
  \subfigure[ Algoritmo.\label{fig:pri2}]{\includegraphics[scale=0.4]{figure/placeholder.jpg}}
  \subfigure[Comportamento.\label{fig:seg2}]{\includegraphics[scale=.4]{figure/placeholder.jpg}}
  \caption{Resultado da atividade prática 1.2, \cite{alguem2022nada}}\label{fig:ap1_cod_vigual1}
\end{figure}

%%%%%%%%%%%%%%%%%%%%%%%%%%%%%%%%%%%%%%%%%%%%%%%%%%%%%%%%%%


Para referenciar utilize \cite{ninguem2022curioso}. Também pode ser citado integrada ao texto, de acordo com \citeonline{alguem2022nada}.

Para inserir imagens adicione a figura no diretório \textit{/figure}






\begin{enumerate}[label=\Roman{*}, ref=(\roman{*})]
  \item fsfsdf
  \item kugfhiuh
\end{enumerate}

\begin{asparaenum}
\item Anterior ... \cite{ninguem2022curioso}
\item Próximo ... \label{pl1}
\end{asparaenum}





\section{Conclusões}



  %$X \xLongleftarrow[\text{NATAN}]{\text{OGLIARI}} Y $ %COM TEXTO
	% $\uparrow$ %Seta para Cima
	%$\overleftarrow{NATAN}$
