
\section{Introdução}


\noindent \begin{minipage}[c]{0.6\textwidth}
\par O uso de banco e dados teve seu inicio com o advento da computação, para o armazenar e gerenciar dados, com as necessidades aumentando diversos pesquisadores foram desenvolvendo e aprimorando uma marco importante segundo \cite{histdb:2024}, em meados de 1980 a IBM desenvolve a \textit{Structured Query Language} (SQL), com o passar dos tempo o ícone que represenda um banco de dados é o da figura \ref{fig:icodb}. Seguindo os avanços e nesseciades tecnilogicas temos dois grandes tipos de banco de dados: os \textbf{Relacionais} e os \textbf{Não relacionais}, dos quais o objeto tema desta aula prática seráo \textbf{Não Relacionais}.


\end{minipage}
\begin{minipage}[c]{0.4\textwidth}
  \captionof{figure}{ícone de banco de dados}
  \includegraphics[width=\textwidth]{figure/pngegg.png}
  \label{fig:icodb}
  {\fontsize{10pt}{\baselineskip}\selectfont
    Fonte: \citeonline{DB:2024}}
\end{minipage}


\section{Métodos}

\begin{algorithm}[H]
  \SetAlgoLined %APAGAR
  \KwData{Entrada do algoritimo}
  \KwIn{entrada} \
   \KwResult{Resultado do codigo} \
   \While{$x = 0$}{
    Leia atual \;
    \eIf{$n = 2$}{
     vá para aproxima seção \;
     a seção atual se torna esta \;
     }{
     VOlta ao inicio da seção \;
      \Return{EXIT}
    }
   }
   \caption{Nome do algoritimo em Portugues}
   {\fontsize{10pt}{\baselineskip}\selectfont
   Fonte: O autor (2022)}

\end{algorithm}





\section{Resultados}

\lstinputlisting[language=c, caption={código externo}, label={cod:externo}, captionpos=t]{main.c} %Busca os codigos na pasta /cod


\subsection{subfiguras}

\begin{figure}[H] %Figuras da aula pratica 1.1
  \center
  \subfigure[ Algoritmo.\label{fig:pri2}]{\includegraphics[scale=0.4]{figure/placeholder.jpg}}
  \subfigure[Comportamento.\label{fig:seg2}]{\includegraphics[scale=.4]{figure/placeholder.jpg}}
  \caption{Resultado da atividade prática 1.2, \cite{oliveira_SO2009}}\label{fig:ap1_cod_vigual1}
\end{figure}

%%%%%%%%%%%%%%%%%%%%%%%%%%%%%%%%%%%%%%%%%%%%%%%%%%%%%%%%%%


Para referenciar utilize \cite{ninguem2022curioso}. Também pode ser citado integrada ao texto, de acordo com \citeonline{alguem2022nada}.

Para inserir imagens adicione a figura no diretório \textit{/figure}






\begin{enumerate}[label=\Roman{*}, ref=(\roman{*})]
  \item fsfsdf
  \item kugfhiuh
\end{enumerate}

\begin{asparaenum}
\item Anterior ... \cite{ninguem2022curioso}
\item Próximo ... \label{pl1}
\end{asparaenum}





\section{Conclusões}



  %$X \xLongleftarrow[\text{NATAN}]{\text{OGLIARI}} Y $ %COM TEXTO
	% $\uparrow$ %Seta para Cima
	%$\overleftarrow{NATAN}$
