
\section{Introdução}




\noindent \begin{minipage}[c]{0.6\textwidth}
  \vspace {1cm}
  \begin{description}
    \item [Banana] Exemplo de mini página com figura e seus respectivos rotulos, para que sejam referenciados ao decorrer do texto.
    \item [Maça] Veja que a Figura \ref{fig:place}, está reservando um espaço para adição de figuras, e o mesmo já esta referenciando seu autor e sua nomeclatura com o indice automatico.
  \end{description}

\end{minipage}
\begin{minipage}[c]{0.4\textwidth}
  \captionof{figure}{Placeholder.}
  \includegraphics[width=\textwidth]{figure/placeholder.jpg}
  \label{fig:place}
  {\fontsize{10pt}{\baselineskip}\selectfont
    Fonte: \citeonline{linux:2023}}
\end{minipage}


\section{Desenvolvimento}

\begin{algorithm}[H]
  \SetAlgoLined %APAGAR
  \KwData{Entrada do algoritimo}
  \KwIn{entrada} \
   \KwResult{Resultado do codigo} \
   \While{$x = 0$}{
    Leia atual \;
    \eIf{$n = 2$}{
     vá para aproxima seção \;
     a seção atual se torna esta \;
     }{
     VOlta ao inicio da seção \;
      \Return{EXIT}
    }
   }
   \caption{Nome do algoritimo em Portugues}
   {\fontsize{10pt}{\baselineskip}\selectfont
   Fonte: O autor (2022)}

\end{algorithm}
\subsection{Método}
\lipsum[2-2]


\subsection{Resultados}
\lipsum[2-2]




\section{Código externo no main.c}

\lstinputlisting[language=c, caption={código externo}, label={cod:externo}, captionpos=t]{main.c} %Busca os codigos na pasta /cod


\section{Sub Figuras}

\begin{figure}[H] %Figuras da aula pratica 1.1
  \center
  \subfigure[ Algoritmo.\label{fig:pri2}]{\includegraphics[scale=0.4]{figure/placeholder.jpg}}
  \subfigure[Comportamento.\label{fig:seg2}]{\includegraphics[scale=.4]{figure/placeholder.jpg}}
  \caption{Resultado da atividade prática 1.2, \cite{oliveira_SO2009}}\label{fig:ap1_cod_vigual1}
\end{figure}

%%%%%%%%%%%%%%%%%%%%%%%%%%%%%%%%%%%%%%%%%%%%%%%%%%%%%%%%%%
\section{Seção que será apagada}

Para referenciar utilize \cite{ninguem2022curioso}. Também pode ser citado integrada ao texto, de acordo com \citeonline{alguem2022nada}.

Para inserir imagens adicione a figura no diretório \textit{/figure}




\section{Sub itens}

\begin{enumerate}[label=\Roman{*}, ref=(\roman{*})]
  \item fsfsdf
  \item kugfhiuh
\end{enumerate}

\begin{asparaenum}
\item Anterior ... \cite{ninguem2022curioso}
\item Próximo ... \label{pl1}
\end{asparaenum}





\section{Conclusões}



  %$X \xLongleftarrow[\text{NATAN}]{\text{OGLIARI}} Y $ %COM TEXTO
	% $\uparrow$ %Seta para Cima
	%$\overleftarrow{NATAN}$
